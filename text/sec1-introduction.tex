\section{Introduction}
\label{sec:introduction}

OxCSProject is a (hopefully) easy to use \LaTeX\ template that you can use to develop and format
your 3\textsuperscript{rd} or 4\textsuperscript{th} year \acrfull{cs} project at the University of
Oxford. This should allow you to avoid all of that silly formatting nonsense and stick to the
thing that you're really here for, your project.

This template comes as the full package, and even compiles to its own user manual, which can also
be found alongside the template on GitHub~\cite{stevensonNedStevensonOxCSProject2024} once you
inevitably start writing up your own work. However, that doesn't mean you shouldn't modify,
extend, and augment the class to fit your needs. That might mean a project for a degree that
\textit{isn't} \acrlong{cs}, updating it to match new formatting requirements, or using your own
set of packages and tools to fit your project.

If you do adapt this template to suit another purpose, such as a project for another Oxford
degree, I encourage you to publish and advertise it just like I do here.

\subsection{Motivation}

I wanted to make this template because navigating \LaTeX\ and trying to adapt and modify John's
template while I was conducting my project was one more thing that I needed to do. It was
certainly worth it, it's something I realised that I could do for your project too, as well as
ahead of my own Part C project. I'm hoping that making it easy to use \LaTeX\ in your project will
mean you choose to do so, making your project better than it otherwise would have been and helping
you develop yet another skill for your future endeavours.

\subsection{Structure of the Document}
The rest of this document is structured as follows:

Section~\ref{sec:installation} describes how to install this template to
OverLeaf~\cite{OverleafOnlineLaTeX}, an easy to use online \LaTeX\ editor. It also includes how to
install this template so that you can develop your project writeup offline if you prefer.

Section~\ref{sec:background} discusses the reasons for working with \LaTeX\ over methods of
developing your project writeup, such as Microsoft Word or Google Docs.

Section~\ref{sec:usage} demonstrates how to use some of the features of \LaTeX\ you may find helpful
to be familiar with while preparing your writeup.

Section~\ref{sec:other_tools} introduces some of the other tools that it's worth becoming
acquainted with to make your project as good as it can be.

Section~\ref{sec:future_work} explains how you, dear reader, can add to this template for the
benefit of anyone who might embark on a project in future.
