\section{Other tools}
\label{sec:other_tools}

This section describes the range of tools aside from \LaTeX\ that you may want to be familiar with
while you're working on your project. This is not intended to be a comprehensive list and is only
the tools that I found useful in my project. Alternatives of course exist to everything I talk
about here, and you should investigate what works for you before simply copying my method.

\subsection{Matplotlib}

Matplotlib~\cite{MatplotlibVisualizationPython} is a Python library that provides easy to use
graph plotting functionality, along side other visualisation features that you may find helpful.
By producing your graph plots in Python, you can integrate it easily into the data processing
stage of your project, if one exists.

This lets you be very versatile with how you display your data, without needing to work within the
more constrained options of something like Microsoft Excel. I also found it more intuitive to
understand the relationships I'm trying to demonstrate when working in Python compared to a
spreadsheet.

See Figure~\ref{fig:stack_overflow} for an example of a plot made using Matplotlib, and see
appendix~\ref{sec:matplotlib_example} for source code of the plot.

\subsection{Zotero}

Zotero~\cite{ZoteroYourPersonal} is a reference manager that allows you to easily collect and keep
track of all the references that you'll need to include in your project writeup. By using the
Zotero browser extension~\cite{ZoteroConnectors} and the Better BibTeX
add-on~\cite{BetterBibTeXZotero}, you can make it easy to add to your bibliography with the
browser extension. The Better BibTeX add-on will then make sure that the citation keys that are
exported from Zotero are consistent and do not clash, making it easy to see what you're
referencing when writing your \LaTeX.

If you use this method, make sure you set the format to Better BibLaTeX when exporting to your
\verb|references.bib| file (or whatever name you end up using for your bibliography file).

\subsection{Draw.io}

Draw.io~\cite{Drawio} is a tool used to draw whatever diagrams you need to explain the core
concepts of your project. Maybe it's important to explain an inheritance diagram, relational
database, or just a simple flowchart for your code.

You can either use the online editor or download the draw.io app to use locally. You can then
export the diagrams you create to a bitmap image format, such as \verb|.png|, or to a vector
format, such as \verb|.svg|. I recommend keeping your draw.io save file in the folder with your
figures whether or not you're using an OverLeaf project.