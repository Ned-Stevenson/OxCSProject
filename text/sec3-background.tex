\section{Background}
\label{sec:background}

\subsection{Benefits of using LaTeX}

\LaTeX\ brings all of the benefits of software development, particularly the ability to reuse work
(both your own and other people's) through the use of subroutines and packages. In \LaTeX, packages
are widely available for many different uses, some of which I try to demonstrate in
Section~\ref{sec:usage}. They are imported through the use of the \verb*|\usepackage| command, and
allow you to subsequently use the commands that they define. You can define your own subroutines
using the \verb*|\newcommand| command, helping you stay on top of keeping your documents concise
and fast to work on.

Another benefit that \LaTeX\ affords us is the separation of our work into multiple, more easily
navigated files. I found it helpful to separate my project to include one file per section, to
keep any one file where I was actually working and writing content to a manageable length. Of
course you don't need to copy me if you prefer to write your documents in a single monolithic
file, but at least you have the option.

\subsection{Challenges of working in conventional word processors}

One of the challenges of working in a word processor is the way that it can slow down as the size
of the document you're navigating grows. This can get quite frustrating, particularly as you try
to get the finishing touches on your formatting the night you're trying to submit your writeup.

Another challenge is working with automatically generated components of your document, such as
bibliographies, tables of contents, and glossaries. In \LaTeX, this type of thing can be neatly
tracked for you by whichever package you're using. And often formatting can be as simple as a
sequence of command arguments, rather than doing it manually after regenerating your tables.

And not least, displaying your code, something I hope you'll do a fair amount, feels far more
challenging in a word processor than in \LaTeX. Your only real option is to copy from your
\acrshort{ide} and keep formatting, which doesn't let you elegantly deal with long lines of code,
change your formatting after the fact, or keep your code in its source files and import it into
your document directly. All these things are handled much more elegantly in \LaTeX, as will be
demonstrated in Section~\ref{sec:code_listing}.
